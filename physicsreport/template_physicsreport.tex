\documentclass[dvipdfmx]{jsarticle}
%%%%%%%%%%%%%%%%%%%%%%%%%%%%%%%%%%%%%%%%%%%%%%%%%%%%%%%%%%%%%%%%%%%%%%%%%%%%%%%%
%%%%    Notes                                                               %%%%
%%%%%%%%%%%%%%%%%%%%%%%%%%%%%%%%%%%%%%%%%%%%%%%%%%%%%%%%%%%%%%%%%%%%%%%%%%%%%%%%

%% 1    Write people's names in katakana, Japanese names in kanji.

%% 2    Don't use \qty of siunitx, use \SI.
%% 2    \qty of physics package can be used as usual.




%%%%%%%%%%%%%%%%%%%%%%%%%%%%%%%%%%%%%%%%%%%%%%%%%%%%%%%%%%%%%%%%%%%%%%%%%%%%%%%%
%%%%    Packages and librarys                                               %%%%
%%%%%%%%%%%%%%%%%%%%%%%%%%%%%%%%%%%%%%%%%%%%%%%%%%%%%%%%%%%%%%%%%%%%%%%%%%%%%%%%

\usepackage{amsmath, amsfonts, amssymb, amsthm}
\usepackage{bm}
\usepackage{color}
\usepackage{comment}
\usepackage{framed}
\usepackage{graphicx}
\usepackage[colorlinks=true, linkcolor=blue, citecolor=blue, urlcolor=blue]{hyperref}
\usepackage{pxjahyper}

%%  Resolve \qty conflict between siunitx and physics package
\usepackage{siunitx}
\usepackage{physics}

%%  Use Times New Roman
%%  \usepackage{newtxtext, newtxmath}
%%  \usepackage{type1cm}




%%%%%%%%%%%%%%%%%%%%%%%%%%%%%%%%%%%%%%%%%%%%%%%%%%%%%%%%%%%%%%%%%%%%%%%%%%%%%%%%
%%%%    Settings of style                                                   %%%%
%%%%%%%%%%%%%%%%%%%%%%%%%%%%%%%%%%%%%%%%%%%%%%%%%%%%%%%%%%%%%%%%%%%%%%%%%%%%%%%%

%%  Format of enumerate environment
\renewcommand{\labelenumi}{(\arabic{enumi})}

%%  Numbering
\counterwithin{figure}{section}
\counterwithin{table}{section}
\numberwithin{equation}{section}

%%  Format of footnote
\renewcommand{\thefootnote}{*\arabic{footnote}}




%%%%%%%%%%%%%%%%%%%%%%%%%%%%%%%%%%%%%%%%%%%%%%%%%%%%%%%%%%%%%%%%%%%%%%%%%%%%%%%%
%%%%    Other Commands                                                      %%%%
%%%%%%%%%%%%%%%%%%%%%%%%%%%%%%%%%%%%%%%%%%%%%%%%%%%%%%%%%%%%%%%%%%%%%%%%%%%%%%%%

%% Reference between parenthesis
\newcommand{\refpr}[1]{(\ref{#1})}

%%  Constant
\newcommand{\const}{\mathrm{const.}}




\pagestyle{myheadings}
\title{physicsreport のテンプレート}
\author{tar2chicken}
\date{\today}


\begin{document}
\maketitle
\tableofcontents
\section{目的}


\section{原理}
	\subsection{現象の原理}
	\subsection{解析の原理}


\section{実験方法}
	\subsection{測定方法と処理方法}
	\subsection{解析方法}


\section{結果}
	\subsection{測定結果}
	\subsection{処理結果}
	\subsection{解析結果}


\section{考察}


\section{付録}
	\subsection{付録}
		付録.
		\clearpage
		\begin{table}[htb]
			\caption{ミラー指数の 2 次形の比}
			\centering
			\begin{tabular}{cccc}
				\hline
				単純 & 体心 & 面心 & ダイヤモンド \\
				\hline
				1 & 1 & 1 & 1 \\
				2 & 2 & 4/3 & 8/3 \\
				3 & 3 & 8/3 & 11/3 \\
				4 & 4 & 11/3 & 16/3 \\
				5 & 5 & 4 & \\
				6 & 6 & 16/3 & \\
				8 & 7 & & \\
				9 & 8 & & \\
				\hline
			\end{tabular}
		\end{table}
		\begin{comment}
			\begin{figure}[htb]
				\centering
				\includegraphics[keepaspectratio, width=40em]{figure/A_raw.jpg}
				\caption{試料 A の測定結果}
			\end{figure}
		\end{comment}
		\addcontentsline{toc}{section}{参考文献}
		\begin{thebibliography}{9}
			\bibitem{author} 著者 ( 著 ), 訳者 ( 訳 ), 『本の名前』 ( シリーズ ), 出版社, 出版年.
		\end{thebibliography}


\end{document}